\documentclass{ctexart}

\usepackage{graphicx}
\usepackage{booktabs}
\usepackage{makecell}
\usepackage{listings}
\usepackage{xcolor}

\usepackage{zhnumber}
\renewcommand{\thesection}{\zhnum{section}}
\renewcommand{\thesubsection}{\arabic{subsection}}

\newcommand\ful[2][4cm]{\underline{\makebox[#1][c]{#2}}}

\usepackage[top=2.54cm, bottom=2.54cm, left=3.18cm, right=3.18cm]{geometry}

\title{\textbf{数据信息安全课程设计 \\ 基于AES的时间锁加密系统设计}}
\author{
    \textbf{姓名:}傅子骏
    \textbf{学号:}2021316101119
}
\date{\today}

\begin{document}

\maketitle

\tableofcontents

\newpage

\section{理论分析}

\subsection{介绍}

时间锁加密(Time-Locked Encryption)是将信息保密到指定时间后才能解密的一种方法。
常见的TLE加密手段包括分布式密钥以及配合区块链智能合约实现。
前者可以通过Shamir门限方案,将密钥分割成n份,分发到服务器保存,只有到达门限k时,才能推出密钥;
后者使用ElGamal算法,使用公钥加密信息,并将私钥发送给以太坊合约,使得区块链矿工可以通过挖矿找到对应的私钥,
通过设置谜题难度控制求解私钥的时间。

\subsection{Sharmir门限方案}
时间服务器TS通过NTP协议访问可信主机,并获取当前准确时间,同时TS保存有部分私钥,这部分私钥被称为子密钥,通过Shamir门限方案,密钥s被分为n份毫不相关的部分信息,只有当解密者同时拥有k份子密钥时,才能恢复出密文s。这种方案被称为(k, n)-秘密分割门限方案。

\subsection{分割AES密钥}

本次课设我将分发n份AES子密钥,只有当解密者同时拥有全部密钥,才能恢复密钥。本次课设我将使用AES128,ECB模式进行加密数据,AES128意味着密钥有16 bytes,通过建立n个服务器,将密钥分成n份,每份(16 / n) bytes,这里假设服务器数量不超过16,对于无法除尽的情况,最后一台服务器保存剩下的bytes。

\subsection{密钥分发}

\subsubsection{RSA算法}

在生产实践中,为了保证双向通信的安全,应当使用RSA算法在两端分别生成私钥和公钥,双方交换公钥,双方生成随机数m、n,
各自对随机数加密并交换加密数据,双方使用私钥验证,验证通过应当交换AES密钥,以增加通信的效率。

\subsubsection{Bcrypt算法}

用户密码往往有规律,如被有心人掌握很容易通过掩码攻击等方式破解,如果使用密钥衍生技术生成强度更高的密码,以此作为AES的密钥将会是很好的做法。

\subsection{通信协议}

由于分发密钥需要保证可靠性、完整性,我们应当选择TCP/IP协议作为密钥分发的通信协议。为了减少依赖数量,提升编译速度,我们自己设计通信协议。

\newpage
\subsubsection{分发密钥协议}

\begin{table}[htbp]
\centering
\caption{分发密钥字节流}
\label{tab:byte_stream}
\begin{tabular}{lc}
\toprule
\textbf{描述} & \textbf{字节} \\
\midrule
OPCODE & + \\
包编号 & 1 \\
AES子密钥长度 & 2 \\
AES子密钥 & 3——3 + 子密钥长度 \\
MD5摘要 & 3 + 子密钥长度——19 + 子密钥长度 \\
时间戳 & 19 + 子密钥长度——最后 \\
\bottomrule
\end{tabular}
\end{table}

时间服务器响应的OPCODE为"+"或"-",前者表示收录成功,后者表示操作失败。

\begin{table}[htbp]
\centering
\caption{分发密钥相应字节流}
\label{tab:byte_stream}
\begin{tabular}{lc}
\toprule
\textbf{描述} & \textbf{字节} \\
\midrule
OPCODE & +/- \\
\bottomrule
\end{tabular}
\end{table}

\subsubsection{请求AES密钥}
\begin{table}[htbp]
\centering
\caption{请求密钥字节流}
\label{tab:byte_stream}
\begin{tabular}{lc}
\toprule
\textbf{描述} & \textbf{字节} \\
\midrule
OPCODE & - \\
时间戳 & 2——最后 \\
\bottomrule
\end{tabular}
\end{table}

请求AES密钥相应字节流的OPCODE同上,如果OPCODE为"+",则第二字节为AES子密钥长度,剩下的字节为AES子密钥。

\newpage
\section{编程实现}

\subsection{准备}

\subsubsection{Rust}

在编程语言的选择上,使用了本人比较喜欢的语言Rust。Rust 速度惊人且内存利用率极高。由于没有运行时和垃圾回收,它能够胜任对性能要求特别高的服务,可以在嵌入式设备上运行,还能轻松和其他语言集成。Rust 丰富的类型系统和所有权模型保证了内存安全和线程安全,让我在编译期就能够消除各种各样的错误。Rust 拥有出色的文档、友好的编译器和清晰的错误提示信息, 还集成了一流的工具——包管理器和构建工具, 智能地自动补全和类型检验的多编辑器支持, 以及自动格式化代码等等。

\subsection{Cli}

为了方便使用期间,我们通过clap构建Cli。Cli是命令行参数解析工具(Command Line Argument Parser for Rust)。目前可用的Cli命令如下,使用cargo run -- --help打印所有可用选项。

\begin{lstlisting}
Usage: tle.exe <COMMAND>

Commands:
  server  Start a server
  client  Start a client
  help    Print this message or the help of the given subcommand(s)

Options:
  -h, --help     Print help
  -V, --version  Print version
\end{lstlisting}

本次课设重心在于客户端的加密与解密操作。
加密时指定输入文件、密码与服务器,通过-s参数可指定多组服务器;
解密时指定输入文件与服务器,若解密条件已达成则解密成功。
客户端的help打印如下。

\begin{lstlisting}
Usage: tle.exe client <COMMAND>

Commands:
  encrypt
  decrypt
  help     Print this message or the help of the given subcommand(s)

Options:
  -h, --help  Print help
\end{lstlisting}

客户端可以使用如下命令加密数据并分发到指定服务器。
\begin{center}
cargo run -- encrypt -i <INPUT> -s <IP:PORT> -p <PASSWORD> -t <TIMESTAMP>
\end{center}

到达约定解密时间后通过运行如下命令解密。
\begin{center}
cargo run -- decrypt -i <INPUT> -s <IP:PORT>
\end{center}

我们使用Rust过程宏生成Cli所需要的参数。本程序接受第一个子参数为选择类型,服务端和客户端均可由一个程序开启,选择客户端时,可选操作有加密和解密。

\subsection{可信NTP服务器}

我们选择阿里云作为可信NTP服务器。通过发起NTP请求获取准确时间。模块定义于src/time.rs。需要注意的是,NTP返回的时间戳为NTP时间戳,NTP时间戳的起始时间为1900-01-01,需要转换到UNIX时间戳,通过减去转换常数获得。

\begin{lstlisting}
use ntp::formats::timestamp::EPOCH_DELTA;
packet.transmit_time.sec as u64 - EPOCH_DELTA as u64
\end{lstlisting}

\subsection{如何标识一个文件}

通过对原文件进行MD5运算,获得对该文件的唯一标识,不仅可以用于服务端检索密钥,也可以用于验证解密后文件的正确性。

\subsection{加密后内容的保存}

文件加载:首先,程序通过load\_file函数加载需要加密的文件内容到内存中。

密码处理:用户输入的密码通过bcrypt\_password函数处理,生成一个用于AES加密的16字节的密钥。

文件内容MD5计算:使用file\_md5函数计算加载的文件内容的MD5散列值。

文件加密:encrypt\_file函数使用AES算法和生成的密钥对文件内容进行加密。这里使用的是ECB模式的AES加密,并且可能使用了PKCS7填充。

保存加密文件:加密后的文件内容与文件的MD5散列值一起被写入到一个新的文件中。新文件的命名是在原始文件名的基础上添加了.tle扩展名。

\newpage
\section{程序测试}

首先我们开启三个终端,两个用于服务器,分别开放端口8001和8002。
命令为
\begin{center}
    cargo run -- server -p 8001 \\
    cargo run -- server -p 8002
\end{center}
\begin{figure}[htbp]
    \centering
    \includegraphics[width=0.8\textwidth]{figures/1.png}
    \caption{开启两个服务}
    \label{fig:t1}
\end{figure}

编写用于加密的文件test.txt,内容如图所示。
\begin{figure}[htbp]
    \centering
    \includegraphics[width=0.8\textwidth]{figures/2.png}
    \caption{加密文件并分发密钥}
    \label{fig:t2}
\end{figure}

\newpage
在规定时间未到达时测试,文件解密失败。
\begin{figure}[htbp]
    \centering
    \includegraphics[width=0.8\textwidth]{figures/3.png}
    \caption{解密失败}
    \label{fig:t3}
\end{figure}

在指定时间之后解密,解密成功,解密得到的文件dec\_test.txt内容与test.txt一致。
\begin{figure}[htbp]
    \centering
    \includegraphics[width=0.8\textwidth]{figures/4.png}
    \caption{解密成功}
    \label{fig:t4}
\end{figure}


\newpage
\section{总结}
本次课设综合AES、MD5、Bcrypt密钥衍生算法、Socket等内容,提出切割AES密钥并保存到服务器,在指定时间之后才能解密的想法,并加以实现,编写过程中收获了很多,例如MD5与hex的转换等,遇到的最大问题在于Socket,约定协议并解析的过程其乐无穷,最终完成本课设。

\newpage
\section{附件}

本次实验所有代码已开源,并放在本人的Github仓库中:https://github.com/Helio609/tle

\end{document}
